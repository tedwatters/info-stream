\documentclass[a4paper, 11pt]{report}
\usepackage{comment} % enables the use of multi-line comments (\ifx \fi) 
\usepackage{lipsum} %This package just generates Lorem Ipsum filler text. 
\usepackage{fullpage} % changes the margin
\usepackage{placeins}

\let\Oldsection\section
\renewcommand{\section}{\FloatBarrier\Oldsection}

\let\Oldsubsection\subsection
\renewcommand{\subsection}{\FloatBarrier\Oldsubsection}

\let\Oldsubsubsection\subsubsection
\renewcommand{\subsubsection}{\FloatBarrier\Oldsubsubsection}

\usepackage[section]{placeins}
\usepackage[a4paper, total={7in, 10in}]{geometry}
\usepackage[fleqn]{amsmath}
\usepackage{amssymb,amsthm}  % assumes amsmath package installed
\newtheorem{theorem}{Theorem}
\newtheorem{corollary}{Corollary}
\usepackage{graphicx}
\usepackage{grffile}
\usepackage[maxfloats=256]{morefloats}
\maxdeadcycles=1000
\usepackage{tikz}
\usepackage{bm}
\usetikzlibrary{arrows}
\usepackage{verbatim}
\usepackage{listings}
%\usepackage[numbered]{mcode}
\usepackage{float}
\usepackage{tikz}
\usetikzlibrary{shapes,arrows}
\usetikzlibrary{arrows,calc,positioning}
\usetikzlibrary{arrows.meta}
\usepackage[
backend=biber,
style=ieee,
sorting=none
]{biblatex}
\addbibresource{bibliography.bib}

\tikzset{
	block/.style = {draw, rectangle,
		minimum height=1cm,
		minimum width=1.5cm},
	input/.style = {coordinate,node distance=1cm},
	output/.style = {coordinate,node distance=4cm},
	arrow/.style={draw, -latex,node distance=2cm},
	pinstyle/.style = {pin edge={latex-, black,node distance=2cm}},
	sum/.style = {draw, circle, node distance=1cm},
}
\usepackage{xcolor}
\usepackage{mdframed}
%\usepackage[shortlabels]{enumitem}
\usepackage{enumitem}
\usepackage{setspace} % for \onehalfspacing and \singlespacing macros
\onehalfspacing 
\usepackage{etoolbox}
\AtBeginEnvironment{quote}{\par\singlespacing\small}
%\usepackage{indentfirst}
\setlength{\parindent}{4em}
\setlength{\parskip}{1em}
\allowdisplaybreaks
\usepackage{hyperref}


\renewcommand{\thesubsection}{\thesection.\alph{subsection}}


\renewcommand{\qed}{\quad\qedsymbol}
%%%%%%%%%%%%%%%%%%%%%%%%%%%%%%%%%%%%%%%%%%%%%%%%%%%%%%%%%%%%%%%%%%%%%%%%%%%%%%%%%%%%%%%%%%%%%%%%%%%%%%%%%%%%%%%%%%%%%%%%%%%%%%%%%%%%%%%%

\title{Inferring trader information stream classification using chain graphs}
\author{Ted Watters}
\begin{document}
\maketitle


\begin{abstract}
	Create a synthetic data set from an agent-based model using NetLogo. Then, conduct chain graph analysis to see how accurately we can infer the number of trader classification categories, and what their probabilities are, given observed trading activity and information streams. As an extra, run a Monte Carlo simulation using the chain grain and then compare the original synthetic data set to the Markov Chain Monte Carlo results.
\end{abstract}\
\tableofcontents
\chapter{Introduction}
\section{The Market}
Consider the following agent-based market model, based on Cont's work~\cite{ghoulmie_cont_nadal_2005}. 

\begin{quote}
	Our model describes a market where a single asset, whose price is denoted by $p_{t}$ , is traded by
	$N$ agents. Trading takes place at discrete periods $t = 0, 1, 2,\dots$. We will see that, provided
	the parameters of the model are chosen in a certain range, we will be able to interpret these
	periods as "trading days". At each period, agents have the possibility of sending an order to
	the market for buying or selling a unit of asset: denoting by $\phi_{i}(t)$ the demand of the agent,
	we have $\phi_{i}(t) = 1$ for a buy order and $\phi_{i}(t) = -1$. We allow the value $\phi_{i}(t)$ to be zero; the
	agent is then inactive at period $t$. The inflow of public information is modeled by a sequence
	of IID Gaussian random variables $(\epsilon_{t},t = 0, 1, 2,\dots)$ with $t \sim N(0, D^{2})$.
\end{quote}
Diverging from this paper, in the model of this market, consider two independent public information streams $\epsilon_{main,t}$ and $\epsilon_{reddit,t}$, rather than a single information stream $\epsilon_{t}$.
\begin{quote}
	The signal $\epsilon_{t}$ [or in this case, $\epsilon_{main,t}$ and $\epsilon_{reddit,t}$] is a forecast of
	the future return $r_{t}$ and each agent has to decide whether the information conveyed by $\epsilon_{t}$ is
	significant, in which case she will place a buy or sell an order according to the sign of$\epsilon_{t}$.
\end{quote}
Also in a divergence from the paper, the trading threshold is not a function of time:
\begin{quote}
	The trading rule of each agent $i = 1,..., N$ is represented by a \dots decision
	threshold [ $\theta_{i} \sim U(0,1)$]. The threshold [ $\theta_{i}$ ] can be viewed as the agent’s (subjective) view on volatility.
	The trading rule we study may be seen as a stylized example of threshold behaviour: without
	sufficient external stimulus [$(|\epsilon_{t}| \leq \theta_{i})$], an agent remains inactive $\phi_{i}(t) = 0$.
\end{quote}
In this case, since we have two information streams $\epsilon_{main,t}$ and $\epsilon_{reddit,t}$ we must define how an agent consumes them. Consider two classes of agents:
\begin{itemize}
	\item [$\rho_{i} = 0$] Consumes only main information stream
	\item [$\rho_{i} = 1$] Consumes both the main and reddit information streams, with a single set importance factor $\beta \in [0,1]$ determining the relative weight of each information stream to the agent's decision making. 
\end{itemize}
The class of each agent does not vary with time, and is randomly determined, based on a single set reddit proportion factor $\alpha \in [0,1]$. Thus:
\begin{equation}\label{trading_decision}
\phi_{i}(t) = \begin{cases}
	1 & \rho_{i}=0,\epsilon_{main,t}>\theta_{i} \\
	1 & \rho_{i}=1,(\beta \epsilon_{reddit,t} + (1-\beta)\epsilon_{main,t}) > \theta_{i}\\
	0 & \rho_{i}=0, |\epsilon_{main,t}| \leq \theta_{i} \\
	0 & \rho_{i}=1, |\beta \epsilon_{reddit,t} + (1-\beta)\epsilon_{main,t}| \leq \theta_{i} \\
	-1 & \rho_{i}=0,\epsilon_{main,t}<\theta_{i} \\
	-1 & \rho_{i}=1,(\beta \epsilon_{reddit,t} + (1-\beta)\epsilon_{main,t}) < \theta_{i}\\
\end{cases}
\end{equation}
\begin{quote}
	The excess demand is then given by $Z_{t} = \sum_{i=1}^{N} \phi_{i}(t)$
\end{quote}
In this market, consider asset price at time $t$ defined by:
\begin{equation}\label{price_equation}
	p_{t} = p_{t-1} \exp \left( \frac{ Z_{t}}{\lambda N}\right)
\end{equation}
Where the sum of trader actions $Z_{t}$ is clearly calculated before $p_{t}$, and $\lambda > 0$ is the normalized market depth such that $g'(0) = \frac{1}{\lambda}$. 

Thus, the general procedure (after market initialization) at each step $t$ is:
\begin{enumerate}
	\item Each trader $i$ makes a buy/ no-action/ sell decision $\phi_{i}(t)$, which is a function of agent class $\rho_{i}$ and information stream $\epsilon_{main,t}$ (and possibly importance factor $\beta$ and $\epsilon_{reddit,t}$).
	\item Price is updated based on previous price $p_{t-1}$, sum of trader actions $Z_{t}$, and 
\end{enumerate}
\section{The Agent Based Model}\label{agent_based_model}
	Full code for the NetLogo~\cite{netlogo} (version 6.2.2) model is available at \url{https://github.com/tedwatters/info-stream/}. This repository is private for employer sensitivity reasons; however, please email \url{ted.watters@gmail.com} for access. Entire code file are included in the appendix \autoref{append:netlogo}, with snippets of interest in the main body of this report.
	
	The general procedure:
	\begin{lstlisting}
		to go
			set vol 0
			create-information
			ask traders [set done? false]
			while [count traders with [not done?] > 0]
			[ask traders [trade-2]]
			calculate-price
			calculate-return
			set vol sum [abs my-action] of traders
			if not write-output? [do-plots]
			if write-output? [do-output]
			tick
			if ticks > maxTicks [set write-output? false file-close-all stop]
		end
	\end{lstlisting}
	
	\noindent For the information streams, the main information stream is a random walk with resets if the information stream walks more than 120\% of the set standard deviation from 0. On the other hand, the reddit information stream has a random walk component, but also takes negative information from the main information stream, to provide contrarian context. It, too, resets if it gets too far from 0.
	
	\begin{lstlisting}
	to create-information
		ifelse not newInfo?
		[
			ifelse paper?
			;; paper case
			[set current-info random-normal 0 (d ^ 2)
			]
	
			[set current-info random-normal 0 d] ;; original line
			]
		[
		ifelse ticks < 2
			[
			set current-reddit random-normal 0 (d ^ 2)
			set current-info random-normal 0 (d ^ 2)
			]
		[
		set current-reddit max list (min list (current-reddit - 0.3 * current-info + random-normal 0 (d ^ 2)) (d * 2)) (d * -0.3)
		ifelse abs(current-info) > (d * 1.2)
			[
			set current-info -0.01
			]
			[
			set current-info current-info + random-normal 0 (d ^ 2)
			]
		]
	]
	end
	\end{lstlisting}
	
	\noindent Trade decisions are made by the following code, which matches the logic shown in ~\autoref{trading_decision}:
	\begin{lstlisting}
	to trade-2
		;; for now only looking at paper case
		ifelse useBothFeeds? and myRedditUser
		[
		ifelse ((redditImportance * current-reddit) + ((1 - redditImportance) * current-info)) < (-1 * my-threshold) or ((redditImportance * current-reddit) + ((1 - redditImportance) * current-info)) > my-threshold
		[
		ifelse ((redditImportance * current-reddit) + ((1 - redditImportance) * current-info)) > 0
		[set my-action 1 set z z + 1][set my-action -1 set z z - 1]
		]
		[set my-action 0]
		]
		[
		;; paper case
		ifelse current-info < (-1 * my-threshold) or current-info > my-threshold
		[
		ifelse current-info > 0
		[set my-action 1 set z z + 1][set my-action -1 set z z - 1]
		]
		[set my-action 0]
		]
		set done? true
	end
	\end{lstlisting}
	
	\noindent And to calculate the updated price, the logic matches ~\autoref{price_equation}:
	\begin{lstlisting}
	to calculate-price
		set previous-price current-price
		set g ((z / (count traders)) / lamda)
		set current-price exp g * previous-price
	end
	\end{lstlisting}
	
	\noindent This model produces a time series that includes:
	\begin{itemize}
		\item $phi_{i}(t)$, trader $i$ buy/ no-action/ sell
		\item $\sum_{i=1}^{N} | \phi_{i}(t) |$, the volume of traders wanting to trade
		\item $Z_{t}$, excess demand
		\item $p_{t}$, price
		\item $\epsilon_{main,t}$ and $\epsilon_{reddit,t}$, the info streams
	\end{itemize}

	For example, see ~\autoref{ex_price}, ~\autoref{ex_vol}, and ~\autoref{ex_info}. The agent based model includes 200 traders, a 2,000 tick run length, a single random starting seed, and no burn in. In some graphs, the abbreviation "RR" refers to the Reddit Ratio, $\alpha$. Similarly, "RI" refers to the Reddit Importance, $\beta$. 
\begin{figure}[h!]
	\caption{Current Price}
	\label{ex_price}
	\includegraphics[width=\textwidth]{../plots/priceRR0-RI0.1.png}
\end{figure}

\begin{figure}[h!]
	\caption{Volume}
	\label{ex_vol}
	\includegraphics[width=\textwidth]{../plots/volRR0-RI0.1.png}
\end{figure}

\begin{figure}[h!]
	\caption{Info Streams}
	\label{ex_info}
	\includegraphics[width=\textwidth]{../plots/infoRR0-RI0.1.png}
\end{figure}

\chapter{Bayesian Inference}
\section{Supporting Research}
	This project was inspired by Epstein~\cite{epstein2007generative} \textit{Chapter 8: The Emergence of Class in a Multi-Agent Bargaining Model}, where it was shown that:
	\begin{quote}
		...We have argued that various kinds of social orders— including segregated, discriminatory, and class systems— can also arise through the decentralized interactions of many agents in which accidents of history become reinforced over time. In these path-dependent dynamics, society may self-organize around distinctions that are quite arbitrary from an a priori standpoint. Above, initially meaningless “tags” acquire socially organizing salience: tag-based classes emerge.
	\end{quote}


	To support this, it is assumed that the structure is known, and that the focus is mostly on parameter estimation. After considerable research, the R causact package seemed to be a good fit ~\cite{causact}, ~\cite{fleischhacker2022}, ~\cite{golding2022}. It provided great visuals, while enabling the powerful greta package, which uses Hamiltonian Markov Chain Monte Carlo (MCMC) ~\cite{radford2010} for inference:
	\begin{quote}
		Hamiltonian dynamics can be used to produce distant proposals for the Metropolis algorithm, thereby avoiding the slow exploration of the state space that results from the diffusive behaviour of simple random-walk proposals. Though originating in physics, Hamiltonian dynamics can be applied to most problems with continuous state spaces by simply introducing fictitious "momentum" variables. A key to its usefulness is that Hamiltonian dynamics preserves volume, and its trajectories can thus be used to define complex mappings without the need to account for a hard-to-compute Jacobian factor - a property that can be exactly maintained even when the dynamics is approximated by discretizing time.
	\end{quote}
	
	One main issue that is encountered with Hamiltonian MCMC is that it only works for continuous state spaces. In this case, while our end inference targets of $\alpha$ and $\beta$ are continuous, certain observed variables (like $\phi_{i}(t)$) and latent variables (like $\rho_{i}$) are discrete in this model. To get around this limitation, for modeling purposes these were redefined:
	\begin{itemize}
		\item $\rho_{i} \in Bernouilli(\alpha) \Longrightarrow \rho_{i} \in \alpha N(1,0.5) + (1-\alpha)N(0,0.5)$\\
		\item $\phi_{i}(t) \in \lbrace -1, 0, 1 \rbrace \Longrightarrow \phi_{i}(t) \in \sum_{j=1}^{m} w_{j} N(\mu_{j},0.5)$, where $\mathbf{w}$ is a $m$ length vector consisting of near-binary values and $\boldsymbol{\mu} = (-1,0,1)^{T}$.\\
	\end{itemize}
	
	These multi-modal distributions can be challenging for learning, because of multiple local maximums.
	
	Additional references that were used for background research include ~\cite{koller2009}, ~\cite{coursera}, ~\cite{sun2013}, ~\cite{abramson2016}, ~\cite{lafferty2001}, ~\cite{thai2018}, ~\cite{neogi2019}, ~\cite{ling2019},  ~\cite{scutari2021},  ~\cite{guo2021},  ~\cite{suter2021}, and ~\cite{mohammadi2021}. These references were summarized in the 50\% submission.
	
\section{The Template Model}
This agent based model can be represented by a template chain graph shown in ~\autoref{chain_graph}. In this case, there are two main plates. The first contains node specific to an agent $i$, and the second refers to nodes specific to time $t$. The gray nodes represent latent variables that we cannot observe, whereas the transparent nodes are ones we can observe. 

\begin{figure}[h]
	\centering
	\begin{tikzpicture}[
	squarednode/.style={rectangle, draw=black!60, fill=black!5, very thick, minimum size=7mm},
	resultnode/.style={rectangle, draw=black!60, very thick, minimum size=7mm},	
	scale = 0.2
	]
	%Nodes
	\node[squarednode, align=center]      (lambda)                              {$\lambda$};	
	\node[squarednode, align=center]      (importance)     [right=of lambda]                        {$\beta$};
	\node[squarednode, align=center]      (proportion)     [left=of lambda]                          {$\alpha$};		
	\node[resultnode, align=center]        (price)       [below=of lambda] {$p_{t}$};		
	\node[resultnode, align=center]        (action)       [below =of price] {$\phi_{i}(t)$};				
	\node[squarednode, align=center]        (class)       [left=of action] {$\rho_{i}$};
	\node[resultnode, align=center]        (main)       [right=of price] {$\epsilon_{main,t}$};				
	\node[resultnode, align=center]        (reddit)       [right=of main] {$\epsilon_{reddit,t}$};
	\node[squarednode, align=center]      (threshold)     [below=of class]      {$\theta_{i}$};					
%	%Lines
	\draw[-{Latex[width=3mm]}] (importance.south) -- (action.north);
	\draw[-{Latex[width=3mm]}] (proportion.south) -- (class.north) ;
	\draw[-{Latex[width=3mm]}] (class.east) -- (action.west);
	\draw[-{Latex[width=3mm]}] (main.west) -- (action.east);
	\draw[-{Latex[width=3mm]}] (reddit.south) -- (action.east);
	\draw[-{Latex[width=3mm]}] (lambda.south) -- (price.north);
	\draw[-{Latex[width=3mm]}] (threshold.north) -- (action.south);
	\draw[-{Latex[width=3mm]}] (action.north) -- (price.south);
	\draw [->] (price.90) arc (0:264:20mm);
%	%rectangle
	\draw[red,thick,dotted] ($(price.north west)+(-2,2)$)  rectangle ($(reddit.south east)+(2,-12)$);
	\draw[blue,thick,dotted] ($(class.north west)+(-2,2)$)  rectangle ($(action.south east)+(2,-12)$);

	\end{tikzpicture}\\
	\caption{Template Chain Graph}
	\label{chain_graph}
\end{figure}

Our primary objectives are to accurately estimate $\alpha \in [0,1]$ and $\beta \in [0,1]$ only given $p_{t}$, $\epsilon_{main,t}$, $\epsilon_{reddit,t}$, and $\phi_{i}(t)$.

\section{Inference Approach}

After transforming the data generated from the agent based model into a format that can be used for this analysis (code available in ~\autoref{append:r}), a graph representation can be created. One of the struggles here is that for the causact package, text substitutions are needed because it does not allow for all greta function parameters to be directly modified.
\begin{lstlisting}[language=R]
graph = dag_create() %>%
    dag_node("Trader Decision","phi",
             rhs = decision_function(rho[ip],betar,epsilon_reddit[tp],epsilon_main[tp],theta[ip]),
             data = trader_decisions$value
             ) %>%
    dag_node(descr = "Trader Classification",label = "rho",
             rhs = normal(15,15),
             child = "phi") %>%
    dag_node(descr = "Proportion of Reddit Traders",label = "alpha",
             rhs = uniform(0,1),
             child = "rho") %>%
    dag_node(descr = "Trader Threshold", label = "theta",
             rhs = beta(2,2),
             child = "phi") %>%
    dag_node(descr =  "Reddit Importance Factor",label = "betar",
             rhs = uniform(0,1),
             child = "phi") %>%
    dag_node("Info Main","epsilon_main",
             data = trader_decisions$current.info,
             child = "phi") %>%
    dag_node("Info Reddit","epsilon_reddit",
             data = trader_decisions$current.reddit,
             child = "phi") %>%
    dag_node("Tick","t",
             data = trader_decisions$ticks,
             child="phi") %>%
    dag_node("Trader","i",
             data = as.numeric(trader_decisions$variable),
             child="phi") %>%
    dag_plate("Trader","ip",
              nodeLabels = c("rho","theta","i","phi"),
              data = trader_decisions$variable) %>%
    dag_plate("Tick Plate","tp",
              nodeLabels = c("epsilon_reddit","epsilon_main","t","phi"),
              data = trader_decisions$tick)

  gretaCode = graph %>% dag_greta(mcmc=FALSE)

  # Needed to set the number of runs from default
  gretaCode <- str_replace(
    gretaCode, 
    "mcmc\\(gretaModel\\)", 
    "mcmc\\(gretaModel, n_samples = 1000, warmup=25\\)"
  )
  
  
  # Needed to make mixture distribution for rho
  gretaCode <- str_replace(
    gretaCode,
    "normal\\(mean = 15, sd = 15, dim = ip_dim\\)", 
    "mixture\\(normal\\(0, 0.5\\), normal\\(1, 0.5\\), weights = c\\(1 - alpha, alpha\\), dim = ip_dim\\)"
  )
  
  # Needed to delete redundant operation and likelihood line
  gretaCode <- str_replace(
    gretaCode,
    "phi    <- decision_function\\(rho_val = rho\\[ip\\], betar_val = betar, er_val = epsilon_reddit\\[tp\\], em_val = epsilon_main\\[tp\\], theta_val = theta\\[ip\\]\\)", 
    ""
  )

  eval(parse(text=gretaCode))
\end{lstlisting}
This code generates the following graphical representation ~\autoref{ex_graph}. Note that the distribution for "rho" is what it was before the mixture substitution was applied.
\begin{figure}[h!]
	\caption{Plate Graph}
	\label{ex_graph}
	\includegraphics[width=\textwidth]{../plots/graphRR0-RI0.1.png}
\end{figure}

\noindent Dark green nodes are observed, and the square node is a decision node

\noindent The modified greta code that is generated:

\begin{lstlisting}[language=R]
## The below greta code will return a posterior distribution 
## for the given DAG. Either copy and paste this code to use greta
## directly, evaluate the output object using 'eval', or 
## or (preferably) use dag_greta(mcmc=TRUE) to return a data frame of
## the posterior distribution: 
epsilon_main <- as_data(trader_decisions$current.info)       #DATA
epsilon_reddit <- as_data(trader_decisions$current.reddit)   #DATA
t <- as_data(trader_decisions$ticks)                         #DATA
i <- as_data(as.numeric(trader_decisions$variable))          #DATA
phi <- as_data(trader_decisions$value)                       #DATA
ip     <- as.factor(trader_decisions$variable)   #DIM
tp     <- as.factor(trader_decisions$tick)   #DIM
ip_dim <- length(unique(ip))   #DIM
tp_dim <- length(unique(tp))   #DIM
alpha  <- uniform(min = 0, max = 1)                    #PRIOR
theta  <- beta(shape1 = 2, shape2 = 2, dim = ip_dim)   #PRIOR
betar  <- uniform(min = 0, max = 1)                    #PRIOR
rho    <- mixture(normal(0, 0.5), normal(1, 0.5), weights = c(1 - alpha, alpha), dim = ip_dim)     #PRIOR
distribution(phi) <- decision_function(rho_val = rho[ip], betar_val = betar, er_val = epsilon_reddit[tp], em_val = epsilon_main[tp], theta_val = theta[ip])   #LIKELIHOOD
gretaModel  <- model(rho,alpha,theta,betar)   #MODEL
meaningfulLabels(graph)
draws       <- mcmc(gretaModel, n_samples = 1000, warmup=25)              #POSTERIOR
drawsDF     <- replaceLabels(draws) %>% as.matrix() %>%
                dplyr::as_tibble()           #POSTERIOR
tidyDrawsDF <- drawsDF %>% addPriorGroups()  #POSTERIOR
\end{lstlisting}

\begin{figure}[h!]
	\caption{Greta Plate Graph}
	\label{greta_graph}
	\includegraphics[width=\textwidth]{../plots/GretaGraphRR0-RI0.1.png}
\end{figure}

This code then outputs confidence intervals for each parameter ~\autoref{ex_paramEst}. Of interest are the "alpha" and "betar" parameters. It also can generate ~\autoref{greta_graph}, which is much more complicated than ~\autoref{ex_graph} because of the mixture distributions and calculations located in the "decision function," $\phi_{i}(t)$.

\section{Inference Results}

\begin{figure}[h!]
	\caption{Parameter Estimate}
	\label{ex_paramEst}
	\includegraphics[width=\textwidth]{../plots/paramEstRR0-RI0.1.png}
\end{figure}

\noindent Observations from this example (~\autoref{ex_paramEst}, $\alpha = 0$,$\beta = 0.1$)
\begin{enumerate}
\item $\rho_{i}$, the graph in the upper left, concentrates around 0, as expected. Again, this is modeled as continuous, even though it should be discrete, because of Hamiltonian Monte Carlo limitations.
\item $\theta_{i} \in U(0,1)$, the graph in the bottom left, is reasonably well estimated. We do not see the entire range, but that may be expected from the limited data points generated from the agent based model and only 1,000 samples being generated during MCMC inference. With the given parameters, even 1,000 samples takes nearly 20 minutes to run for each scenario, because of the complex relationships shown in ~\autoref{greta_graph}.
\item $\alpha$ is estimated to be low, as expected.
\item $\beta$ has a wide range of uncertainty. This makes sense, as $\beta$ has no effect when $\alpha = 0$, so no learning can take place.
\end{enumerate}

\noindent In ~\autoref{append:fig}, runs from 9 scenarios are included. 

\noindent The summary from these 9 runs are in ~\autoref{alpha_summary} and ~\autoref{beta_summary}



\begin{figure}[h!] 
	\caption{$\alpha$ Summary}
	\label{alpha_summary}
	 \includegraphics[width=\textwidth]{../plots/alphasummary.png} 
\end{figure}
\begin{figure}[h!] 
	\caption{$\beta$ Summary} 
	\label{beta_summary}
	\includegraphics[width=\textwidth]{../plots/betasummary.png} 
\end{figure}

\noindent Observations from this summary ~\autoref{alpha_summary}, ~\autoref{beta_summary}.
\begin{enumerate}
\item The trend shows that, in general, higher RR ($\alpha$) scenarios have higher estimated $\alpha$.
\item For high RR scenarios, the predicted $\alpha$ is much lower than expected. In ~\autoref{future_work}, some opportunities for parameter estimation are suggested.
\item Similar observations are noted for RI ($\beta$).
\end{enumerate}

\chapter{Data Generation Comparison: ABM vs. MCMC}

\section{Introduction}
Synthetic time series data sets can be generated by Agent Based Models (ABM), as shown in ~\autoref{agent_based_model}. They can also be generated by Markov Chain Monte Carlo (MCMC). The purpose of this section is to attempt to replicate the ABM in MCMC, and compare bulk measures of the results.

The code for this is somewhat lengthy, and no individual code segment concisely describes the loop. However, the code is available in the repository and is well commented. Key notes about this implementation:
\begin{enumerate}
\item In this application, a modified approach to Gibbs sampling was used. In particular, some of the conditional probabilities were stochastic, while others where deterministic.
\item As written, the mcmc.R script takes approximately 1 hour to run. Longer runs would allow for more data, and may improve the likelihood we are seeing a limiting distribution.
\item Because the reddit information stream gets input from the main information stream, then the transition probability matrix for the reddit information stream depends on time, so it is not ergodic. Thus, we may not expect to get a limiting distribution.
\item For the ABM, a single starting point for the information streams was used. In the MCMC, the starting point was chosen based on the most stable trace from the initial short runs. This likely causes the MCMC time series to be more stable than the ABM time series.
\item The ABM starts every run from the same random seed, so the only thing that varies between them are the scenario parameters ($\alpha$, $\beta$). On the other hand, the MCMC script includes a random seed, but each run is done in series, building off of each other, adding more variance than is seen in the ABM time series.
\end{enumerate}
 
\section{Results}

First, we can review some of the MCMC plots showing trace parameters.

\begin{figure}[h!] 
\caption{Trace: $\alpha = 0$, $\beta=0.1$}
\label{ex_trace}
\includegraphics[width=\textwidth]{../plots/trace00.1.png} 
\end{figure}

\begin{figure}[h!] 
\caption{Cumulative Sum: $\alpha = 0$, $\beta=0.1$}
\label{ex_sum}
 \includegraphics[width=\textwidth]{../plots/sum00.1.png} 
\end{figure}

\begin{figure}[h!] 
\caption{Autocorrelation: $\alpha = 0$, $\beta=0.1$, Starting Point 1}
\label{ex_acf}
 \includegraphics[width=\textwidth]{../plots/acf00.11.png} 
\end{figure}

\noindent Observations:
\begin{enumerate}
\item All traces in ~\autoref{ex_trace} and ~\autoref{ex_sum} are in black. In the code, if the mean of the trace is more than 20\% away from the starting point, then the trace is shown as red, for being unstable. 
\item There does not appear to be much mixing/ wiggling.
\item As expected for a random walk, each observation is highly correlated with the values before it, as seen in ~\autoref{ex_acf}.
\end{enumerate}

\noindent Comparing the ABM and MCMC data series histograms:

\begin{figure}[h!] 
\caption{MCMC Histogram: $\alpha = 0$, $\beta=0.1$ }
\label{ex_mcmchist}
 \includegraphics[width=\textwidth]{../plots/histogramMCMC00.1.png} 
\end{figure}

\begin{figure}[h!] \caption{ABM Histogram: $\alpha = 0$, $\beta=0.1$ }
\label{ex_abmhist} \includegraphics[width=\textwidth]{../plots/histogramABMRR0-RI0.1.png} 
\end{figure}

\noindent Observations:
\begin{enumerate}
\item The data in ~\autoref{ex_abmhist} and ~\autoref{ex_mcmchist} appear to have similar variance, and possibly several modes, depending on bin choice.
\item The means are different, and significant since one is below the common price starting point of 100 and the other is over.
\end{enumerate}

\noindent We have similar, and possibly better, performance at high $\alpha$ and $\beta$ parameter choices in ~\autoref{ex_mcmchist_high} and ~\autoref{ex_abmhist_high}.

\begin{figure}[h!] 
\caption{MCMC Histogram: $\alpha = 0.6$, $\beta=0.7$ }
\label{ex_mcmchist_high}
 \includegraphics[width=\textwidth]{../plots/histogramMCMC0.60.7.png} 
\end{figure}

\begin{figure}[h!] \caption{ABM Histogram: $\alpha = 0.6$, $\beta=0.7$ }
\label{ex_abmhist_high} \includegraphics[width=\textwidth]{../plots/histogramABMRR0.6-RI0.7.png} 
\end{figure}

\noindent A full list of histograms are shown in ~\autoref{append:fig}.

\chapter{Future Work}\label{future_work}

\begin{enumerate}
\item Improved computational efficiency. As written, the code is slow to run. Optimizations can be made such that more data can be generated. Additionally, leveraging cloud computing could improve performance.
\item Explore other inference methods. Greta allows for optimization. This was tested during this project, but the default options provided similar results to the MCMC inference method.
\item Changes to continuous mixture distributions, which attempt to model discrete distributions. These multi-modal distributions are challenging to perform inference with. Besides changing MCMC parameters, updates to the mixture distributions could improve performance. For instance, a standard deviation of 0.5 was selected for the $N(0,0.5)$ and $N(1,0.5)$ distributions. This ensures that there is some overlap between the two distributions, which ensures there is a gradient between the two modes. Smaller or larger standard deviations could be tested.
\item Changes to Hamiltonian MCMC parameters. Starting point selection can have a significant impact on MCMC performance. Additionally, there may be other parameters that can be altered directly in greta that would allow for improved mixing.
\item Consider using importance sampling. Many trader/ tick combinations result in no trading activity. Those points provide may provide little information, and preferentially sampling points that have trading activity may allow for quicker model convergence.
\end{enumerate}

\chapter{Bibliography}
\printbibliography

\chapter{Appendix}
\section{Figures}\label{append:fig}
%"for f in ./plots/*.png; do echo $( printf "\\\begin{figure}[h] \n\\caption{File: .$f} \n \\includegraphics[width=\\\textwidth]{.$f} \n \\\end{figure} \n"); done;"

\subsection{Parameter Inference Summary}
\begin{figure}[h!] \caption{File: ../plots/alphasummary.png} \includegraphics[width=\textwidth]{../plots/alphasummary.png} \end{figure}
\begin{figure}[h!] \caption{File: ../plots/betasummary.png} \includegraphics[width=\textwidth]{../plots/betasummary.png} \end{figure}

\subsection{Inference Graphs}
\begin{figure}[h!] \caption{File: ../plots/graphRR0-RI0.1.png} \includegraphics[width=\textwidth]{../plots/graphRR0-RI0.1.png} \end{figure}
\begin{figure}[h!] \caption{File: ../plots/GretaGraphRR0-RI0.1.png} \includegraphics[width=\textwidth]{../plots/GretaGraphRR0-RI0.1.png} \end{figure}

\subsection{MCMC Autocorrelation}
\begin{figure}[h!] \caption{File: ../plots/acf00.11.png} \includegraphics[width=\textwidth]{../plots/acf00.11.png} \end{figure}
\begin{figure}[h!] \caption{File: ../plots/acf001.png} \includegraphics[width=\textwidth]{../plots/acf001.png} \end{figure}
\begin{figure}[h!] \caption{File: ../plots/acf00.71.png} \includegraphics[width=\textwidth]{../plots/acf00.71.png} \end{figure}
\begin{figure}[h!] \caption{File: ../plots/acf0.10.11.png} \includegraphics[width=\textwidth]{../plots/acf0.10.11.png} \end{figure}
\begin{figure}[h!] \caption{File: ../plots/acf0.101.png} \includegraphics[width=\textwidth]{../plots/acf0.101.png} \end{figure}
\begin{figure}[h!] \caption{File: ../plots/acf0.10.71.png} \includegraphics[width=\textwidth]{../plots/acf0.10.71.png} \end{figure}
\begin{figure}[h!] \caption{File: ../plots/acf0.60.11.png} \includegraphics[width=\textwidth]{../plots/acf0.60.11.png} \end{figure}
\begin{figure}[h!] \caption{File: ../plots/acf0.601.png} \includegraphics[width=\textwidth]{../plots/acf0.601.png} \end{figure}
\begin{figure}[h!] \caption{File: ../plots/acf0.60.71.png} \includegraphics[width=\textwidth]{../plots/acf0.60.71.png} \end{figure}

\subsection{NetLogo Pricing Histograms}
\begin{figure}[h!] \caption{File: ../plots/histogramABMRR0.1-RI0.1.png} \includegraphics[width=\textwidth]{../plots/histogramABMRR0.1-RI0.1.png} \end{figure}
\begin{figure}[h!] \caption{File: ../plots/histogramABMRR0.1-RI0.7.png} \includegraphics[width=\textwidth]{../plots/histogramABMRR0.1-RI0.7.png} \end{figure}
\begin{figure}[h!] \caption{File: ../plots/histogramABMRR0.1-RI0.png} \includegraphics[width=\textwidth]{../plots/histogramABMRR0.1-RI0.png} \end{figure}
\begin{figure}[h!] \caption{File: ../plots/histogramABMRR0.6-RI0.1.png} \includegraphics[width=\textwidth]{../plots/histogramABMRR0.6-RI0.1.png} \end{figure}
\begin{figure}[h!] \caption{File: ../plots/histogramABMRR0.6-RI0.7.png} \includegraphics[width=\textwidth]{../plots/histogramABMRR0.6-RI0.7.png} \end{figure}
\begin{figure}[h!] \caption{File: ../plots/histogramABMRR0.6-RI0.png} \includegraphics[width=\textwidth]{../plots/histogramABMRR0.6-RI0.png} \end{figure}
\begin{figure}[h!] \caption{File: ../plots/histogramABMRR0-RI0.1.png} \includegraphics[width=\textwidth]{../plots/histogramABMRR0-RI0.1.png} \end{figure}
\begin{figure}[h!] \caption{File: ../plots/histogramABMRR0-RI0.7.png} \includegraphics[width=\textwidth]{../plots/histogramABMRR0-RI0.7.png} \end{figure}
\begin{figure}[h!] \caption{File: ../plots/histogramABMRR0-RI0.png} \includegraphics[width=\textwidth]{../plots/histogramABMRR0-RI0.png} \end{figure}

\subsection{MCMC Pricing Histograms}
\begin{figure}[h!] \caption{File: ../plots/histogramMCMC00.1.png} \includegraphics[width=\textwidth]{../plots/histogramMCMC00.1.png} \end{figure}
\begin{figure}[h!] \caption{File: ../plots/histogramMCMC00.7.png} \includegraphics[width=\textwidth]{../plots/histogramMCMC00.7.png} \end{figure}
\begin{figure}[h!] \caption{File: ../plots/histogramMCMC00.png} \includegraphics[width=\textwidth]{../plots/histogramMCMC00.png} \end{figure}
\begin{figure}[h!] \caption{File: ../plots/histogramMCMC0.10.1.png} \includegraphics[width=\textwidth]{../plots/histogramMCMC0.10.1.png} \end{figure}
\begin{figure}[h!] \caption{File: ../plots/histogramMCMC0.10.7.png} \includegraphics[width=\textwidth]{../plots/histogramMCMC0.10.7.png} \end{figure}
\begin{figure}[h!] \caption{File: ../plots/histogramMCMC0.10.png} \includegraphics[width=\textwidth]{../plots/histogramMCMC0.10.png} \end{figure}
\begin{figure}[h!] \caption{File: ../plots/histogramMCMC0.60.1.png} \includegraphics[width=\textwidth]{../plots/histogramMCMC0.60.1.png} \end{figure}
\begin{figure}[h!] \caption{File: ../plots/histogramMCMC0.60.7.png} \includegraphics[width=\textwidth]{../plots/histogramMCMC0.60.7.png} \end{figure}
\begin{figure}[h!] \caption{File: ../plots/histogramMCMC0.60.png} \includegraphics[width=\textwidth]{../plots/histogramMCMC0.60.png} \end{figure}


\subsection{Info Stream}
\begin{figure}[h!] \caption{File: ../plots/infoRR0-RI0.1.png} \includegraphics[width=\textwidth]{../plots/infoRR0-RI0.1.png} \end{figure}

\subsection{Scenario Parameter Estimates}
\begin{figure}[h!] \caption{File: ../plots/paramEstRR0.1-RI0.1.png} \includegraphics[width=\textwidth]{../plots/paramEstRR0.1-RI0.1.png} \end{figure}
\begin{figure}[h!] \caption{File: ../plots/paramEstRR0.1-RI0.7.png} \includegraphics[width=\textwidth]{../plots/paramEstRR0.1-RI0.7.png} \end{figure}
\begin{figure}[h!] \caption{File: ../plots/paramEstRR0.1-RI0.png} \includegraphics[width=\textwidth]{../plots/paramEstRR0.1-RI0.png} \end{figure}
\begin{figure}[h!] \caption{File: ../plots/paramEstRR0.6-RI0.1.png} \includegraphics[width=\textwidth]{../plots/paramEstRR0.6-RI0.1.png} \end{figure}
\begin{figure}[h!] \caption{File: ../plots/paramEstRR0.6-RI0.7.png} \includegraphics[width=\textwidth]{../plots/paramEstRR0.6-RI0.7.png} \end{figure}
\begin{figure}[h!] \caption{File: ../plots/paramEstRR0.6-RI0.png} \includegraphics[width=\textwidth]{../plots/paramEstRR0.6-RI0.png} \end{figure}
\begin{figure}[h!] \caption{File: ../plots/paramEstRR0-RI0.1.png} \includegraphics[width=\textwidth]{../plots/paramEstRR0-RI0.1.png} \end{figure}
\begin{figure}[h!] \caption{File: ../plots/paramEstRR0-RI0.7.png} \includegraphics[width=\textwidth]{../plots/paramEstRR0-RI0.7.png} \end{figure}
\begin{figure}[h!] \caption{File: ../plots/paramEstRR0-RI0.png} \includegraphics[width=\textwidth]{../plots/paramEstRR0-RI0.png} \end{figure}


\subsection{NetLogo Price}
\begin{figure}[h!] \caption{File: ../plots/priceRR0.1-RI0.1.png} \includegraphics[width=\textwidth]{../plots/priceRR0.1-RI0.1.png} \end{figure}
\begin{figure}[h!] \caption{File: ../plots/priceRR0.1-RI0.7.png} \includegraphics[width=\textwidth]{../plots/priceRR0.1-RI0.7.png} \end{figure}
\begin{figure}[h!] \caption{File: ../plots/priceRR0.1-RI0.png} \includegraphics[width=\textwidth]{../plots/priceRR0.1-RI0.png} \end{figure}
\begin{figure}[h!] \caption{File: ../plots/priceRR0.6-RI0.1.png} \includegraphics[width=\textwidth]{../plots/priceRR0.6-RI0.1.png} \end{figure}
\begin{figure}[h!] \caption{File: ../plots/priceRR0.6-RI0.7.png} \includegraphics[width=\textwidth]{../plots/priceRR0.6-RI0.7.png} \end{figure}
\begin{figure}[h!] \caption{File: ../plots/priceRR0.6-RI0.png} \includegraphics[width=\textwidth]{../plots/priceRR0.6-RI0.png} \end{figure}
\begin{figure}[h!] \caption{File: ../plots/priceRR0-RI0.1.png} \includegraphics[width=\textwidth]{../plots/priceRR0-RI0.1.png} \end{figure}
\begin{figure}[h!] \caption{File: ../plots/priceRR0-RI0.7.png} \includegraphics[width=\textwidth]{../plots/priceRR0-RI0.7.png} \end{figure}
\begin{figure}[h!] \caption{File: ../plots/priceRR0-RI0.png} \includegraphics[width=\textwidth]{../plots/priceRR0-RI0.png} \end{figure}


\subsection{MCMC Cumulative Sum}
\begin{figure}[h!] \caption{File: ../plots/sum00.1.png} \includegraphics[width=\textwidth]{../plots/sum00.1.png} \end{figure}
\begin{figure}[h!] \caption{File: ../plots/sum00.7.png} \includegraphics[width=\textwidth]{../plots/sum00.7.png} \end{figure}
\begin{figure}[h!] \caption{File: ../plots/sum00.png} \includegraphics[width=\textwidth]{../plots/sum00.png} \end{figure}
\begin{figure}[h!] \caption{File: ../plots/sum0.10.1.png} \includegraphics[width=\textwidth]{../plots/sum0.10.1.png} \end{figure}
\begin{figure}[h!] \caption{File: ../plots/sum0.10.7.png} \includegraphics[width=\textwidth]{../plots/sum0.10.7.png} \end{figure}
\begin{figure}[h!] \caption{File: ../plots/sum0.10.png} \includegraphics[width=\textwidth]{../plots/sum0.10.png} \end{figure}
\begin{figure}[h!] \caption{File: ../plots/sum0.60.1.png} \includegraphics[width=\textwidth]{../plots/sum0.60.1.png} \end{figure}
\begin{figure}[h!] \caption{File: ../plots/sum0.60.7.png} \includegraphics[width=\textwidth]{../plots/sum0.60.7.png} \end{figure}
\begin{figure}[h!] \caption{File: ../plots/sum0.60.png} \includegraphics[width=\textwidth]{../plots/sum0.60.png} \end{figure}


\subsection{MCMC Trace}
\begin{figure}[h!] \caption{File: ../plots/trace00.1.png} \includegraphics[width=\textwidth]{../plots/trace00.1.png} \end{figure}
\begin{figure}[h!] \caption{File: ../plots/trace00.7.png} \includegraphics[width=\textwidth]{../plots/trace00.7.png} \end{figure}
\begin{figure}[h!] \caption{File: ../plots/trace00.png} \includegraphics[width=\textwidth]{../plots/trace00.png} \end{figure}
\begin{figure}[h!] \caption{File: ../plots/trace0.10.1.png} \includegraphics[width=\textwidth]{../plots/trace0.10.1.png} \end{figure}
\begin{figure}[h!] \caption{File: ../plots/trace0.10.7.png} \includegraphics[width=\textwidth]{../plots/trace0.10.7.png} \end{figure}
\begin{figure}[h!] \caption{File: ../plots/trace0.10.png} \includegraphics[width=\textwidth]{../plots/trace0.10.png} \end{figure}
\begin{figure}[h!] \caption{File: ../plots/trace0.60.1.png} \includegraphics[width=\textwidth]{../plots/trace0.60.1.png} \end{figure}
\begin{figure}[h!] \caption{File: ../plots/trace0.60.7.png} \includegraphics[width=\textwidth]{../plots/trace0.60.7.png} \end{figure}
\begin{figure}[h!] \caption{File: ../plots/trace0.60.png} \includegraphics[width=\textwidth]{../plots/trace0.60.png} \end{figure}


\subsection{NetLogo Volume}
\begin{figure}[h!] \caption{File: ../plots/volRR0.1-RI0.1.png} \includegraphics[width=\textwidth]{../plots/volRR0.1-RI0.1.png} \end{figure}
\begin{figure}[h!] \caption{File: ../plots/volRR0.1-RI0.7.png} \includegraphics[width=\textwidth]{../plots/volRR0.1-RI0.7.png} \end{figure}
\begin{figure}[h!] \caption{File: ../plots/volRR0.1-RI0.png} \includegraphics[width=\textwidth]{../plots/volRR0.1-RI0.png} \end{figure}
\begin{figure}[h!] \caption{File: ../plots/volRR0.6-RI0.1.png} \includegraphics[width=\textwidth]{../plots/volRR0.6-RI0.1.png} \end{figure}
\begin{figure}[h!] \caption{File: ../plots/volRR0.6-RI0.7.png} \includegraphics[width=\textwidth]{../plots/volRR0.6-RI0.7.png} \end{figure}
\begin{figure}[h!] \caption{File: ../plots/volRR0.6-RI0.png} \includegraphics[width=\textwidth]{../plots/volRR0.6-RI0.png} \end{figure}
\begin{figure}[h!] \caption{File: ../plots/volRR0-RI0.1.png} \includegraphics[width=\textwidth]{../plots/volRR0-RI0.1.png} \end{figure}
\begin{figure}[h!] \caption{File: ../plots/volRR0-RI0.7.png} \includegraphics[width=\textwidth]{../plots/volRR0-RI0.7.png} \end{figure}
\begin{figure}[h!] \caption{File: ../plots/volRR0-RI0.png} \includegraphics[width=\textwidth]{../plots/volRR0-RI0.png} \end{figure}




\section{R Code}[language=R]
\label{append:r}
\lstinputlisting{../chaingraph.R}
\section{NetLogo Code}
\label{append:netlogo}
\lstinputlisting{../market.nlogo}
	

\end{document}          
